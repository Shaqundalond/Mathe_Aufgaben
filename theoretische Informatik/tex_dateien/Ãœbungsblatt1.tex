\documentclass[18pt,a4paper]{article}
\usepackage[T1]{fontenc}
\usepackage{setspace}
\usepackage{amsmath}
\usepackage{amsfonts}

\begin{document}
	\section*{Aufgabe1}
	Es sei:\\
	$A:= \{1,17,53\}$ \\
	$B:= \{2,17,23\}$
	\subsection*{a$)$ }
	Berechne die Vereinigung $(\cup)$ von A und B\\
	$A \cup B = \{1,17,53\} \cup \{2,17,23\} = \{1,2,17,23,53\} $
	\subsection*{b$)$ }
	Berechne den Schnitt $(\cap)$ von A und B\\
	$A\cap B = \{1,17,53\} \cap \{2,17,23\} = \{17\} $
	\subsection*{c$)$ }
	\doublespacing
	Berechne das kartesische Produkt $(\times)$ von A und B\\
	$A\times B = \{1,17,53\} \times \{2,17,23\} = $\\
	$\{(1,2),(1,17),(1,23),(17,2),(17,17),(17,23),(53,2),(53,17),(53,23)\} $\\
	\singlespacing
	\subsection*{d$)$ }
	Berechne die Differenz $(\setminus)$ von A und B\\
	$A\setminus B = \{1,17,53\} \setminus \{2,17,23\} = \{1,53\} $
	
	\section*{Aufgabe2}
	\subsection*{a$)$ }
	$\mathcal{P}(\emptyset) = \{\emptyset \} $\\
	Die Potenzmenge der leeren Menge ist eine Menge welche die leere Menge als Element besitzt.\\
	$\Longrightarrow\lvert\mathcal{P}(\emptyset)\rvert = 1$
	\subsection*{b$)$ }
	$\mathcal{P}(\mathcal{P}(\emptyset))$\\
	Wir ersetzen die innere Potenzmenge nun mit der vorherig berechneten Potenzmenge\\
	In Worten bestimmen wir die Potenzmenge der Ein-elementigen Menge mit der leeren Menge als Element.\\
	$\mathcal{P}(\{\emptyset \}) = \{\emptyset , \{\emptyset \}\} $\\
	$\Longrightarrow \lvert\mathcal{P}(\mathcal{P}(\emptyset))\rvert= \lvert\mathcal{P}(\{\emptyset \})\rvert = 2$\\
	Es gilt im allg.: Die Potenzmenge einer einelementigen Mengen hat zwei Elemente. Einmal die leere Menge und die ganze Menge.\\


	\subsection*{c$)$ }
	$\mathcal{P}(\mathcal{P}(\mathcal{P}(\emptyset)))$\\
	Auch hier ersetzt man die inneren Potenzmengen durch die vorher berechnete Potenzmenge\\
	
	\begin{align*}
	\mathcal{P}(\{\emptyset , \{\emptyset \}\} ) = \{ &\\
	&\emptyset ,\\
	&\{\emptyset\},\{\{\emptyset \}\},\\
	&\{\emptyset , \{\emptyset \}\}\\
	&\}
	\end{align*}
	$\Longrightarrow \lvert \mathcal{P}(\mathcal{P}(\mathcal{P}(\emptyset))) \rvert = \lvert\mathcal{P}(\{\emptyset , \{\emptyset \}\} )\rvert = 4$
	
	\subsection*{Bemerkung}
	Es lohnt sich nicht die Potenzmengenfunktion von au{\ss}en nach innen aufzul\"osen.\\
	Es gilt im allgemeinen \textbf{nicht}: $\mathcal{P}(\mathcal{P}(\mathcal{P}(\emptyset))) = \{\emptyset , \mathcal{P}(\mathcal{P}(\emptyset)) \}$\\
	Hier liegt eine verstecke Annahme drin, dass $\mathcal{P}(\mathcal{P}(\emptyset))$ eine einelementige Menge ist.\\
	Eine g\"ultige Aussage lautet:\\
	$\mathcal{P}(\mathcal{P}(\mathcal{P}(\emptyset))) = \{\emptyset , \mathcal{P}(\mathcal{P}(\emptyset)) \} \cup \{$alle echten Teilmengen von $\mathcal{P}(\mathcal{P}(\emptyset))\setminus\emptyset\}$\\
	Wenn wir dies weiteraufl\"osen wird das ganze sehr lang und kompliziert\\
	$\mathcal{P}(\mathcal{P}(\mathcal{P}(\emptyset))) = \{\emptyset , \{\emptyset , \mathcal{P}(\emptyset)\} \cup \{ $alle echten Teilmengen von $\mathcal{P}(\emptyset)\setminus\emptyset\}\} \cup \{$alle echten Teilmengen von $\mathcal{P}(\mathcal{P}(\emptyset))\setminus\emptyset\}$\\
	
	
	\section*{Aufgabe 3}
	\subsection*{a$)$}
	Bestimme die Kardinalit\"at des Schnitts  $(\cap)$  von $\{x, y, z\}$ und $\{x, \{y, z\}, \{a\}\}$\\
	$\lvert\{x, y, z\} \cap \{x, \{y, z\}, \{a\}\} \rvert = \lvert\{x{}\}\rvert = 1$
	
	\subsection*{b$)$}
	Bestimme die Kardinalit\"at des kartesischen Produktes von $\mathbb{N} \cap \left[3,6 \right)$ und $\{5\}$\\
	\doublespacing
	$\mathbb{N} \cap \left[3,6 \right) = \{3,4,5\}$\\
	$\{3,4,5\} \times \{5\}=\{(3,5),(4,5),(5,5)\}$\\
	$\lvert \{(3,5),(4,5),(5,5)\} \rvert = 3$\\
	Die Menge besteht aus 3 2-Tupeln
	\singlespacing

	\section*{Aufgabe 4}
	\begin{center}
	$M:= \{1,2,3,4,5\}$\newline
	$R: = \{(1,2),(1,4),(2,4),(3,2),(3,4),(3,5),(5,2),(5,3),(5,4) \}$
	\end{center}
	\subsection*{reflexiv/irreflexiv}
	Die Relation ist irreflexiv, da f\"ur alle Elemente m in M $(m,m)$ nicht in der Relation R ist und somit kann diese nicht reflexiv sein. Reflexiv und irreflexiv schlie{\ss}en sich gegenseitig aus. Eine Relation ist entweder reflexiv oder irreflexiv, bzw keines der beide.\\
	\\
	Es gilt: $\forall m \in M : (m,m) \notin R $
	\subsection*{symetrisch}
	Die Relation ist nicht symetrisch da $(2,4) \in R $ gilt, aber $(4,2) \notin R$ ist, was von der Symetrie gefordert ist.\\
	\subsection*{antisymetrisch}
	Die Relation ist nicht antisymetrisch da $\{(3,5),(5,3)\} \subset R$ gilt. Antisymetrie fordert, dass nur eines der beiden 2-Tupel in R liegt.\\
	\subsection*{transitiv}
	Die Relation ist nicht transitiv da $\{(3,5),(5,3)\} \subset R$ gilt aber $(3,3) \notin R$\\
	\subsection*{total}
	Die Relation ist nicht total, da Totalit\"at Reflexivit\"at fordert und die Relation wie oben beschrieben irreflexiv ist.
	
	\section*{Aufgabe 5}
	\begin{center}
		$R:=\{(x,y) \in \mathbb{N}\times\mathbb{N}\vert \exists z \in \mathbb{N}: y=z\cdot x \}$
	\end{center}
	Die Relation ist weder eine Halb- noch Totalordnung da sie nicht antisymetrisch ist.\\
	Weder $(3,2)$ noch $(2,3)$ sind Elemente der Relation und somit ist diese nicht antisymetrisch und damit auch keine Halb- oder Totalordnung, da hier Antisymetrie gefordert wird
		
\end{document}
