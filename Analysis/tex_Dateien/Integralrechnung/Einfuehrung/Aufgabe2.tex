\documentclass[18pt, 4paper]{article}
\usepackage[ngerman]{babel}
\usepackage[T1]{fontenc}

\usepackage[utf8]{inputenc}
\usepackage{setspace}
\usepackage{amsmath}
\usepackage{amsfonts}
\usepackage{verbatim}
\usepackage{fancyhdr}
\usepackage{graphicx}
\usepackage[left=2cm, top=2cm, right=2cm, bottom=1.5cm]{geometry}
%\usepackage{sidecap}

\pagestyle{fancy} %eigener Seitenstil
\fancyhf{} %alle Kopf- und Fußzeilenfelder bereinigen
\fancyhead[L]{Integralrechnung} %Kopfzeile links
\fancyhead[C]{Fabian Pfaff} %zentrierte Kopfzeile
\fancyhead[R]{\today} %Kopfzeile rechts
\fancyfoot[L]{Integralrechnung} %Fußzeile links
\fancyhead[C]{Fabian Pfaff} %zentrierte Kopfzeile
\fancyfoot[C]{\thepage} %Seitennummer
\fancyfoot[R]{\today}
\renewcommand{\footrulewidth}{0.4pt} %untere Trennlinie

\begin{document}
\section*{1 Einführung}
\subsection*{2}
Bestimmen Sie jeweils alle Stammfunktionen
\begin{align*}
f_1(x) &=\frac{2x^3}{3} - \frac{x}{2} + 1 - \frac{3}{2x} + \frac{4}{3x^2}\\
F_1(x) &= \frac{2x^3}{9} - \frac{x^2}{4} + x - \frac{3}{2}\cdot\ln(2x) - \frac{4}{3x}+c\\
\\
f_2(x) &= sin(x) - 2cos(2x) + 3sin(3x) - 4cos(4x)\\
F_2(x) &=  -cos(x) -sin(2x) - cos(3x) - sin(4x) +c\\
\\
f_3(x) &= (1 - 2x)^2 + (1 - 2x)^1 + (1 - 2x)^0 + (1 - 2x)^{-1} + (1 - 2x)^{-2}\\
F_3(x) &=  -\frac{1}{6}\cdot(1-2x)^3 - \frac{1}{4}\cdot(1-2x)^2 + x - \frac{1}{2}\cdot \ln(1-2x) + \frac{1}{2} \cdot (1-2x)^{-1} +c
\end{align*}
\subsection*{3}
Bestimmen Sie die Integrale
\begin{align*}
I_1 &= 
\int_{0}^{\pi/2}cos(x) \mathrm{d}x = 
sin(x)\bigg\vert_{0}^{\pi/2} 
= sin(\pi/2) -sin(0) 
= 1 - 0 
= 1
\\
\\
I_2 &= 
\int_{0}^{\pi}cos(x) \mathrm{d}x = 
sin(x)\bigg\vert_{0}^{\pi} = 
sin(\pi) -sin(0) = 
0 - 0 = 
0
\\
\\
I_3 &= 
\int_{0}^{\pi}sin(2x) \mathrm{d}x = 
-\frac{1}{2}cos(2x)\bigg\vert_{0}^{\pi} = 
-\frac{1}{2}\left(cos(2\pi) - cos(0)\right) = 
-\frac{1}{2}\left(1-1\right)=
0
\\
\\
I_4 &= 
\int_{1}^{2}(1-x)^2 \mathrm{d}x = 
-\frac{1}{3}(1-x)^3\bigg\vert_{1}^{2} = 
-\frac{1}{3}\left((1-2)^3 -(1-1)^3\right) = 
-\frac{1}{3}\left(-1\right)=
\frac{1}{3}
\\
\\
I_5 &=
\int_{1}^{\mathrm{e}}\frac{1}{x} \mathrm{d}x = 
\ln(x)\bigg\vert_{1}^{\mathrm{e}} = 
\ln(e)-ln(1) = 
1-0 = 
1
\\
\\
I_6 &= 
\int_{-\mathrm{e}^2}^{-\frac{1}{\mathrm{e}}}\frac{1}{x} \mathrm{d}x = 
-\int_{\frac{1}{\mathrm{e}}}^{\mathrm{e}^2}\frac{1}{x} \mathrm{d}x =
-\ln(x)\bigg\vert_{\frac{1}{\mathrm{e}}}^{\mathrm{e}^2} = 
-\left(ln(e^2)-ln(\frac{1}{\mathrm{e}})\right) = 
-(2-(-1)) = 
-3
\end{align*}

\end{document}