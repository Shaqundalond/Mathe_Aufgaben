\documentclass[18pt,a4paper]{article}
\usepackage[T1]{fontenc}
\usepackage{setspace}
\usepackage{amsmath}

\usepackage{amsfonts}
\begin{document}

\section*{Aufgabe 9}
Bestimmen Sie $\displaystyle \lim_{n \to \infty}\left( \sqrt{n^2 - n +1} - n \right) $\\
\\
Zun\"achst wird der Ausdruck mit Hilfe der 3. binomischen Formel erweitert. Dies ist die Standardvorgehensweise wenn man h\"assliche Wurzeln vor sich hat.
\begin{align*}
\sqrt{n^2 - n +1} - n &= \sqrt{n^2 - n +1} - n \cdot \frac{\sqrt{n^2 - n +1} + n}{\sqrt{n^2 - n +1} + n}\\[10pt]
&=\frac{\left(\sqrt{n^2 - n +1} - n\right) \cdot \left(\sqrt{n^2 - n +1} + n\right)}{\sqrt{n^2 - n +1} + n}\\[10pt]
&=\frac{n^2 - n + 1 -n^2}{\sqrt{n^2 - n +1} + n}\\[10pt]
&=\frac{- n + 1}{\sqrt{n^2 - n +1} + n}
\intertext{Dieser Ausdruck sieht auf den ersten Blick nicht besser aus wenn man ihn mit $\frac{1}{n}$ erweitert wird das ganze wieder sch\"oner}
&=\frac{(- n + 1 )\cdot \frac{1}{n}}{\left(\sqrt{n^2 - n +1} + n\right) \cdot \frac{1}{n}}\\[10pt]
&=\frac{- 1 + \frac{1}{n}}{ \frac{\sqrt{n^2 - n +1}}{n} + 1}
\intertext{ersetzt man nun das n im Nenner durch $\sqrt{n^2}$ und zieht das in die Wurzel erh\"alt man folgenden Ausdruck}
&=\frac{- 1 + \frac{1}{n}}{\sqrt{\frac{n^2}{n^2} - \frac{n}{n^2} + \frac{1}{n^2}} + 1}\\[10pt]
&=\frac{- 1 + \frac{1}{n}}{\sqrt{1 - \frac{1}{n} + \frac{1}{n^2}} + 1}
\end{align*}
Hiervon kann man nun den Grenzwert bilden\\[10pt]
$\displaystyle \lim_{n \to \infty}\left(\frac{- 1 + \frac{1}{n}}{\sqrt{1 - \frac{1}{n} + \frac{1}{n^2}} + 1}\right) = \frac{-1+0}{\sqrt{1 - 0 + 0}+1} = -\frac{1}{2}$

\end{document}