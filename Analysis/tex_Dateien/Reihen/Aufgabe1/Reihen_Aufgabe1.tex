\documentclass[18pt,a4paper]{article}
\usepackage[T1]{fontenc}
\usepackage{setspace}
\usepackage{amsmath}
\usepackage{amsfonts}
\usepackage[utf8]{inputenc}
\usepackage{fancyhdr}
%\usepackage[yyyymmdd]{datetime}
%\renewcommand{\dateseparator}{}
%\usepackage{datetime2}

\pagestyle{fancy} %eigener Seitenstil
\fancyhf{} %alle Kopf- und Fußzeilenfelder bereinigen
\fancyhead[L]{Reihen} %Kopfzeile links
\fancyhead[C]{Fabian Pfaff} %zentrierte Kopfzeile
\fancyhead[R]{\today} %Kopfzeile rechts

\begin{document}
\subsection*{Aufgabe 1}
Bestimmen Sie die Summe aller ungeraden dreistelligen Zahlen.\\
\begin{equation}
\displaystyle S_n = \sum_{i=1}^{n} a_i = \frac{n}{2}(a_1 +a_n)
\end{equation}
\begin{equation}
a_n = a_1 + (n-1) \cdot d
\end{equation}
Es ist zu sehen, dass die erste dreistellige ungerade Zahl $a_1$ 101 ist und die letze Zahl $a_n$999 sowie dass die Differenz 2 beträgt. Hieraus kann man nun n mittels Gleichung (2) bestimmen.
\begin{equation}
\frac{a_n - a_1}{d} + 1 =n
\end{equation}
Daraus ergibt sich, dass $n=450$ ist.\\
Mittels Gleichung (1)kann man nun die Summe berechnen\\
\begin{equation*}
S_n =\\
\frac{n}{2}(a_1 +a_n)=\\
\frac{450}{2}(101 +999)=\\
247500
\end{equation*}
Die Summe aller ungeraden dreistelligen Zahlen beträgt \underline{247500}

\end{document}