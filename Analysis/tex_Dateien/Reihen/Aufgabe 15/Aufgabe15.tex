\documentclass[18pt,a4paper]{article}
\usepackage[ngerman]{babel}
\usepackage[T1]{fontenc}

\usepackage[utf8]{inputenc}
\usepackage{setspace}
\usepackage{amsmath}
\usepackage{amsfonts}
\usepackage{verbatim}
\usepackage{fancyhdr}
\usepackage{graphicx}
\usepackage[left=2cm, top=3cm, right=2cm, bottom=1.5cm]{geometry}
%\usepackage[yyyymmdd]{datetime}
%\renewcommand{\dateseparator}{}
%\usepackage{datetime2}

\pagestyle{fancy} %eigener Seitenstil
\fancyhf{} %alle Kopf- und Fußzeilenfelder bereinigen
\fancyhead[L]{Reihen} %Kopfzeile links
\fancyhead[C]{Fabian Pfaff} %zentrierte Kopfzeile
\fancyhead[R]{\today} %Kopfzeile rechts
\fancyfoot[L]{Reihen}
\fancyfoot[C]{\thepage} %Seitennummer
\fancyfoot[R]{\today}
\renewcommand{\footrulewidth}{0.4pt} %untere Trennlinie

\begin{document}
\section*{Aufgabe 11b}
Bestimmen Sie die Taylor-Reihe von $f(x)=(x\cdot\ln(x))^2$ um die Stelle $x_0 = 1$ bis zum Grad 2.\\
Bestimmen Sie den Unterschied von $f(x)$ und dieser Näherung and der Stelle $x=1,1$
\begin{equation}
T_2f(x;x_0)= \sum_{n=0}^{2}\frac{f^{(n)}(x_0)}{n!} \cdot (x-x_0)^n \qquad\forall x,x_0 \in \mathbb{C}
\end{equation}
$x_0$ ist hierbei die Entwicklungsstelle. In der Mathemathik wird gerne $x_0$ verwendet um eine beliebige Zahl zu verwenden, die aber konstant bleibt
\subsection*{Ableitungen}
\begin{multline*}
\\
f^{(0)}(x)=(x\cdot\ln(x))^2\\
f^{(0)}(1)= (1\cdot\ln(1))^2 = 0\\
\\
f^{(1)}(x) = 2(x\cdot ln(x))\cdot(\ln(x)+1) = \underline{2x\cdot\ln^2(x)+ 2x\cdot \ln(x)}\\
f^{(1)}(1) = 2\cdot1\cdot\ln^2(1)+ 2\cdot 1\cdot \ln(1) = 0\\
\\
f^{(2)}(x) = 2\cdot\ln^2(x) + 4 x\cdot\ln(x) \cdot \frac{1}{x} + 2\cdot ln(x) + 2 \cdot \frac{x}{x} = \underline{ 2\cdot\ln^2(x) +  6 \cdot\ln(x) + 2}\\
f^{(2)}(1) = 2\cdot\ln^2(1) +  6 \cdot\ln(1) + 2 = 2\\
\end{multline*}
\subsection*{Taylor-Reihe}
Nun setzen wir die Ergebnisse der Ableitungen in Gleichung 1 ein und erhalten folgenden Ausdruck
\begin{equation*}
T_2f(x;1)= \frac{0}{0!}(x-1)^0 + \frac{0}{1!}(x-1)^1 + \frac{2}{2!}(x-1)^2 = (x-1)^2
\end{equation*}
\subsection*{absoluter Fehler}
Berechne$f(1,1) - T_2f(1,1;1)$\\
\begin{multline*}
\\
f(1,1) \approx 0,011\\
T_2f(1,1;1) = 0,01\\
\Rightarrow f(1,1) - T_2f(1,1;1) \approx 0,001\\
\end{multline*}
Dies entspricht einem relativen Fehler von ca 10\%
\end{document}