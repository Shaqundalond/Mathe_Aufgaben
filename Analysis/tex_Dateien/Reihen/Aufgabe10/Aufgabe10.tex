\documentclass[18pt,a4paper]{article}
\usepackage[ngerman]{babel}
\usepackage[T1]{fontenc}

\usepackage[utf8]{inputenc}
\usepackage{setspace}
\usepackage{amsmath}
\usepackage{amsfonts}
\usepackage{verbatim}
\usepackage{fancyhdr}
\usepackage{graphicx}
\usepackage[left=2cm, top=3cm, right=2cm, bottom=1.5cm]{geometry}
%\usepackage[yyyymmdd]{datetime}
%\renewcommand{\dateseparator}{}
%\usepackage{datetime2}

\pagestyle{fancy} %eigener Seitenstil
\fancyhf{} %alle Kopf- und Fußzeilenfelder bereinigen
\fancyhead[L]{Reihen} %Kopfzeile links
\fancyhead[C]{Fabian Pfaff} %zentrierte Kopfzeile
\fancyhead[R]{\today} %Kopfzeile rechts
\fancyfoot[L]{Reihen}
\fancyfoot[C]{\thepage} %Seitennummer
\fancyfoot[R]{\today}
\renewcommand{\footrulewidth}{0.4pt} %untere Trennlinie

\begin{document}
\section*{Aufgabe 10}
Bestimmen Sie einige Summanden der Taylor-Reihe von $f(x)=\sqrt{x} = x^{\frac{1}{2}}$ um die Stelle $x_0 = 1$
\begin{equation}
Tf(x;x_0)= \sum_{n=0}^{\infty}\frac{f^{(n)}(x_0)}{n!} \cdot (x-x_0)^n \qquad\forall x,x_0 \in \mathbb{C}
\end{equation}
$x_0$ ist hierbei die Entwicklungsstelle. In der Mathemathik wird gerne $x_0$ verwendet um eine beliebige Zahl zu verwenden, die aber konstant bleibt
\begin{equation}
\binom{q}{n} = \frac{\prod_{i=0}^{n-1}(q-i)}{n!} \qquad \qquad \forall q \in \mathbb{R},\forall n \in \mathbb{N}
\end{equation}
\subsection*{Ableitungen}
\begin{align*}
&f^{(0)}(x_0) =& &(x_0)^{\frac{1}{2}}
&f^{(0)}(1) =& 1\\
&f^{(1)}(x_0) =& (\frac{1}{2})\cdot&(x_0)^{-\frac{1}{2}}
&f^{(1)}(1) =& (\frac{1}{2})\\
&f^{(2)}(x_0) =& (-\frac{1}{2})\cdot(\frac{1}{2})\cdot&(x_0)^{-\frac{3}{2}}
&f^{(2)}(1) =& (\frac{1}{2})\cdot(-\frac{1}{2})\\
&f^{(3)}(x_0) =& (-\frac{3}{2})\cdot(-\frac{1}{2})\cdot(\frac{1}{2})\cdot&(x_0)^{-\frac{5}{2}}
&f^{(3)}(1) =& (\frac{1}{2})\cdot(-\frac{1}{2})\cdot(-\frac{3}{2})\\
\end{align*}
\subsection*{Taylor-Reihe}
Es ist nun zu sehen dass  $f^{(n)}(1)$ sich durch folgenden Ausdruck berechnen lässt.
\begin{equation}
f^{(n)}(1) = \prod_{i=0}^{n-1}\left(\frac{1}{2}-i\right)
\end{equation}
Setzen wir nun Gleichung 3 in Gleichung 1 ein erhalten wir folgendenden Ausdruck:
\begin{equation}
Tf(x;1)= \sum_{n=0}^{\infty}\frac{\prod_{i=0}^{n-1}\left(\frac{1}{2}-i\right)}{n!} \cdot (x-1)^n
\end{equation}
Hier ist nun zu erkennen dass der Bruch sich mittels Gleichung 2 vereinfachen lässt, zu
\begin{equation*}
Tf(x;1)= \sum_{n=0}^{\infty}\binom{1/2}{n} \cdot (x-1)^n
\end{equation*}
Hierbei handelt es sich um einen ähnlichen Ausdruck für die Funktionswerte wie bei der Mc-Laurin-Entwicklung der Funktion $f(x)= \sqrt{(x+1)}$ um $x_0 = 0$. Der Unterschied besteht nur darin, dass die Funktion auf der x-Achse verschoben wurde. Der Konvergenzradius kann damit direkt übernommen werden, da dieser nur von den Koeffezienten der Potenzreihe abhängt und diese sich bei Verschiebung und anschließender Entwicklungstelleanpassung nicht verändern.
\end{document}