\documentclass[18pt,4paper]{article}
\usepackage[ngerman]{babel}
\usepackage[T1]{fontenc}
\usepackage[utf8]{inputenc}
\usepackage{setspace}
\usepackage{amsmath}
\usepackage{amsfonts}
\usepackage{verbatim}
\usepackage{fancyhdr}
\usepackage{graphicx}
\usepackage[left=2cm,top=2cm,right=2cm,bottom=1.5cm]{geometry}
\usepackage{graphicx}
\usepackage{sidecap}
\usepackage{polynom}
\polyset{style=C}
\pagestyle{fancy}%eigenerSeitenstil
\fancyhf{}%alleKopf-undFußzeilenfelderbereinigen
\fancyhead[L]{Differenzialrechnung-Funktionen}%Kopfzeilelinks
\fancyhead[C]{FabianPfaff}%zentrierteKopfzeile
\fancyhead[R]{\today}%Kopfzeilerechts
\fancyfoot[L]{Differenzialrechnung-Funktionen}%Fußzeilelinks
\fancyhead[C]{FabianPfaff}%zentrierteKopfzeile
\fancyfoot[C]{\thepage}%Seitennummer
\fancyfoot[R]{\today}
\renewcommand{\footrulewidth}{0.4pt}%untereTrennlinie

\begin{document}
\section*{Aufgabe2}
Zerlegen Sie jeweils in Linearfaktoren
\subsection*{b)}
\begin{equation*}
	b(x) = -x^4-\frac{16}{3}x^3-17x^2-12x+\frac{52}{3}
\end{equation*}
Man beachte, dass der erste koeffizient ungleich 1 ist. Nachdem alle Nullstellen berechnet wurden muss man noch mit diesem Koeffizienten multiplizieren um auf die Ausgangsfunktion zu kommen.
\subsubsection*{Polynomdivision}
Für die Polynomdivision teilt man das Polynom durch die bekannten Nullstellen.

\[\polylongdiv{-x^4 - \frac{16}{3}x^3 -17x^2 - 12x+\frac{52}{3}}{x+2}\]
\hfill
\[\polylongdiv{-x^3-\frac{10}{3}x^2-\frac{31}{3}x+\frac{26}{3}}{x-\frac{2}{3}}\]

\subsubsection*{Nullstellen quadratische Gleichung}
Nun müssen nur noch die Nullstellen, mittels pq-Formel,  des oben erhaltenen Ausdrucks bestimmt werden.
\begin{align*}
	x^2 + 4x +13 = 0\\
	\Rightarrow x_{1/2} = -\frac{4}{2} \pm \sqrt{\frac{16}{4}-13}\\
	\quad  x_{1/2} = -2 \pm \sqrt{-9}\\
	x_1=-2+3i\quad x_2 = -2-3i
\end{align*}
\subsubsection*{Linearfaktoren}
Aus allen Nullstellen und dem ersten Vorfaktor ergibt sich folgende Zerlegung:
\begin{equation*}
	b(x)=-(x+2)(x-\frac{2}{3})(x-(-2+3i))(x-(-2-3i))
\end{equation*}

\newpage

\subsection*{c)}
\begin{equation*}
	c(x) = 4x^4-4x^3-17x^2-16x+4
\end{equation*}
Man beachte, dass der erste koeffizient ungleich 1 ist. Nachdem alle Nullstellen berechnet wurden muss man noch mit diesem Koeffizienten multiplizieren um auf die Ausgangsfunktion zu kommen.
\subsubsection*{Polynomdivision}
Für die Polynomdivision teilt man das Polynom durch die bekannten Nullstellen.
\[\polylongdiv{4x^4-4x^3+17x^2-16x+4}{x-\frac{1}{2}}\]
\hfill
\[\polylongdiv{4x^3-2x^2+16x-8}{x-\frac{1}{2}}\]

\subsubsection*{Nullstellen quadratische Gleichung}
Nun werden die restlichen Nullstellen bestimmt.
\begin{align*}
	4x^2+16=0\\
	x^2+4=0\\
	\Rightarrow x_1 = 2i\quad x_2=-2i
\end{align*}
\subsubsection*{Linearfaktoren}
Aus allen Nullstellen und dem ersten Vorfaktor ergibt sich folgende Zerlegung:
\begin{equation*}
	c(x)=4(x-\frac{1}{2})^2(x-2i)(x+2i)
\end{equation*}
\end{document}

