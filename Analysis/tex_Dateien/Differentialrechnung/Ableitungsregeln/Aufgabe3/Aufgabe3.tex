\documentclass[18pt, 4paper]{article}
\usepackage[ngerman]{babel}
\usepackage[T1]{fontenc}

\usepackage[utf8]{inputenc}
\usepackage{amsmath}
\usepackage{amsfonts}
\usepackage{verbatim}
\usepackage{fancyhdr}
\usepackage[left=2cm, top=2cm, right=2cm, bottom=1.5cm]{geometry}

\pagestyle{fancy} %eigener Seitenstil
\fancyhf{} %alle Kopf- und Fußzeilenfelder bereinigen
\fancyhead[L]{Differenzialrechnung-Funktionen} %Kopfzeile links
\fancyhead[C]{Fabian Pfaff} %zentrierte Kopfzeile
\fancyhead[R]{\today} %Kopfzeile rechts
\fancyfoot[L]{Differenzialrechnung-Funktionen} %Fußzeile links
\fancyhead[C]{Fabian Pfaff} %zentrierte Kopfzeile
\fancyfoot[C]{\thepage} %Seitennummer
\fancyfoot[R]{\today}
\renewcommand{\footrulewidth}{0.4pt} %untere Trennlinie

\begin{document}
\section*{Aufgabe 3 Kettenregel}
Bestimmen Sie die Ableitungen der Funktionen.\\
Die Kettenregel lautet:
\begin{equation*}
	\left(u(v)\right)' = u'(v)\cdot v'
\end{equation*}
\begin{align*}
	f_1(x) &= (2-3x)^5\\
	f_1^{'}(x) &= 5(2-3x)^4\cdot3\\
	f_1^{'}(x) &= 15(2-3x)^4\\
\end{align*}
\hfill\\
\begin{align*}
	f_2(x) &= \ln(\sqrt{x})\\
	f_2^{'}(x) &= \frac{1}{\sqrt{x}} \cdot \frac{1}{2\sqrt{x}}\\
	f_2^{'}(x) &= \frac{1}{2x}\\
\end{align*}
\hfill\\
\begin{align*}
	f_3(x) &= \sqrt{\ln(x)}\\
	f_1^{'}(x) &= \frac{1}{2\sqrt{\ln(x)}} \cdot \frac{1}{x}\\
\end{align*}
\begin{align*}
	f_4(x) &= a \cdot \sqrt{b - c\cdot x}= a\cdot(b - c\cdot x)^{\frac{1}{n}}\\
	f_4{'}(x) &= \frac{a}{n} \cdot (b- c\cdot x)^{\frac{1-n}{n}} \cdot (-c)\\
	f_4{'}(x) &= \frac{-ac}{n \cdot (b- c\cdot x)^{\frac{n-1}{n}}}\\
\end{align*}
\begin{align*}
	f_5(x) &= \left(f(x)\right)^n\\
	f_5^{'}(x) &= n \cdot \left(f(x)\right)^{n-1} \cdot f^{'}(x)\\
\end{align*}
\

\end{document}

