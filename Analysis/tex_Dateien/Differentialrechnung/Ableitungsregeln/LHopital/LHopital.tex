\documentclass[18pt, 4paper]{article}
\usepackage[ngerman]{babel}
\usepackage[T1]{fontenc}

\usepackage[utf8]{inputenc}
\usepackage{amsmath}
\usepackage{amsfonts}
\usepackage{mathtools}
\usepackage{verbatim}
\usepackage{fancyhdr}
\usepackage[left=2cm, top=2cm, right=2cm, bottom=1.5cm]{geometry}

\pagestyle{fancy} %eigener Seitenstil
\fancyhf{} %alle Kopf- und Fußzeilenfelder bereinigen
\fancyhead[L]{Differenzialrechnung-Funktionen} %Kopfzeile links
\fancyhead[C]{Fabian Pfaff} %zentrierte Kopfzeile
\fancyhead[R]{\today} %Kopfzeile rechts
\fancyfoot[L]{Differenzialrechnung-Funktionen} %Fußzeile links
\fancyhead[C]{Fabian Pfaff} %zentrierte Kopfzeile
\fancyfoot[C]{\thepage} %Seitennummer
\fancyfoot[R]{\today}
\renewcommand{\footrulewidth}{0.4pt} %untere Trennlinie

\begin{document}
\section*{Aufgabe 5 L'Hôpital}
Bestimmen Sie den Grenzwert mit L'Hôpital\\
Die Regel von L'Hôpital lautet:
\begin{equation*}
	\lim_{x \to x_0}\frac{f(x)}{g(x)} = \lim_{x \to x_0}\frac{f^{'}(x)}{g^{'}(x)}= \dots
\end{equation*}
\begin{align*}
	\quad \lim_{x \to 0}& \frac{\sin(x)}{x}= \frac{0}{0}	&
	\quad \lim_{x \to 0}& \frac{\sin(3x)}{x}= \frac{0}{0}	&
	\quad \lim_{x \to 0}& \frac{\ln(x)}{x^r}= \frac{-\infty}{0}\\
	\xRightarrow{\text{LH}} \lim_{x \to 0}& \frac{\cos(x)}{1}=1	&
	\xRightarrow{\text{LH}} \lim_{x \to 0}& \frac{3\cos(3x)}{1}=3	&
	\xRightarrow{\text{LH}} \lim_{x \to 0}& \frac{\frac{1}{x}}{rx^{r-1}}=\frac{1}{rx^r}=\infty\\
\end{align*}

\begin{align*}
	\lim_{x \to x^+} x^x &= 0^0\\
	\intertext{Aus $0^0$ ist keine Aussage möglich. Wenn wir jedoch die Funktion anders hinschreiben ist eine Aussage möglich}
	x^x &= id \circ x^x\\
	x^x &= f \circ f^{-1} \circ x^x\\
	x^x &= \exp(\ln(x^x))\\
	x^x &= e^{\ln\left(x^x\right)}\\
	x^x &= e^{x\ln(x)}\\
	\intertext{von diesem Ausdruck kann man nun den Grenzwert berechnen}
	\Rightarrow \lim_{x \to x^+}x^x &= \lim_{x \to x^+}e^{x\ln(x)}\\
	\intertext{Da die Exponentialfunktion eine stetige Funktion ist und stetige Funktionen den Grenzwert erhalten, können wir "lim" in den Exponenten schreiben und den Grenzwer von $x\ln(x)$ betrachten}
	\lim_{x \to x^+}e^{x\ln(x)} &= \exp(\lim_{x \to x^+}x\ln(x))\\
	\Rightarrow  \lim_{x \to x^+}x\ln(x) &=  \lim_{x \to x^+}\frac{\ln(x)}{x^{-1}}= \frac{-\infty}{\infty}\\
	\xRightarrow{\text{LH}} \lim_{x \to 0^+}\frac{\frac{1}{x}}{\frac{-1}{x^2}}&=\lim_{x \to 0^+}\frac{x}{-1} = 0\\
	\Rightarrow \lim_{x \to 0^+}x^x &= \lim_{x \to 0^+}e^{x\ln(x)}= e^0 = 1
\end{align*}
\end{document}



